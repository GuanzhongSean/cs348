\documentclass{article}
\usepackage[a4paper,margin=1in]{geometry}
\usepackage{amsmath,amsfonts,amssymb,graphicx}
\title{\textbf{CS 348 Winter 2025: Assignment 1}}
\author{Jiaze Xiao \\ 20933691}
\date{\today}
\setlength{\parindent}{0pt}
\begin{document}

\maketitle

\section*{Question 1}
\subsection*{(a)}
\begin{itemize}
    \item Data model:\\
          A data model is a set of concepts and rules that define the structure and behavior of a database. It specifies the format, types, and relationships of the data stored in the database. A data model is used to organize and manage data, and to ensure data consistency and integrity.
    \item Data-definition language (DDL):\\
          DDL is a language used to define the structure of a database. It includes commands to create, modify, and delete tables, indexes, and constraints. DDL is typically used to define the schema of a database, which is the set of tables, their attributes, and their relationships.
    \item Candidate key:\\
          A candidate key is a set of one or more attributes that can uniquely identify a tuple in a table. A candidate key is chosen to minimize the risk of data anomalies and ensure data integrity. A candidate key is usually defined as a primary key or a unique index.
\end{itemize}

\subsection*{(b)}
Imagine a student's grade record is being updated concurrently by two teachers in different subjects.
A student's grade record in the database contains:
\begin{center}
    \texttt{student = \{"name": "John", "math": 85, "science": 90\}}
\end{center}
Concurrent Execution:
\begin{itemize}
    \item \textbf{Teacher 1}: The teacher in math updates the grade for John in math to 90.
    \item \textbf{Teacher 2}: The teacher in science updates the grade for John in science to 85.
\end{itemize}
If Teacher 1 and Teacher 2 read the record at the same time and each update their respective subject, assume Teacher 1 updates the record first and then Teacher 2 updates the record. The final record is incorrect because it reflects only one teacher's update: \texttt{\{"math": 85, "science": 85\}} instead of \texttt{\{"math": 90, "science": 85\}}. The concurrent updates caused one update to overwrite the other, leading to lost data.

\newpage
\subsection*{(c)}
A foreign key is an attribute (or a set of attributes) in one table that references the primary key in another table. It establishes a relationship between the two tables, ensuring referential integrity. A foreign key constraint enforces referential integrity by ensuring that the value in the foreign key column matches a value in the referenced primary key column or is \texttt{NULL}.

\subsection*{(d)}
\textbf{F}, the set difference operation (\texttt{R - S}) is monotone with respect to \texttt{R} but non-monotone with respect to \texttt{S}. Since monotonicity depends on which base relation is being considered, the claim that set difference is always non-monotone on its base relations is false.

\subsection*{(e)}
\begin{itemize}
    \item When to Use a Foreign Key Constraint and Not a Tuple-Based CHECK:\\
          For referential integrity between tables. For example, an \texttt{Orders} table contains a column \texttt{CustomerID} that should always reference a valid \texttt{CustomerID} in the \texttt{Customers} table. A foreign key constraint ensures that every \texttt{CustomerID} in the \texttt{Orders} table matches an existing \texttt{CustomerID} in the \texttt{Customers} table.
    \item When to Use a Tuple-Based CHECK and Not a Foreign Key Constraint:\\
          For attribute-specific validation. For example, a \texttt{Products} table contains a column \texttt{Price} that should always be greater than or equal to 0. A tuple-based CHECK constraint ensures that every \texttt{Price} in the \texttt{Products} table is greater than or equal to 0.
\end{itemize}

\subsection*{(f)}
$$
    \texttt{COPURCHASES} := \pi_{\texttt{cID1, cID2}} \bigg( \sigma_{\texttt{cID1} \neq \texttt{cID2}} \left( \rho_{(\texttt{cID}\rightarrow\texttt{cID1}, \texttt{oID}\rightarrow\texttt{oID1})} \texttt{Order} \bowtie \rho_{(\texttt{cID}\rightarrow\texttt{cID2}, \texttt{oID}\rightarrow\texttt{oID2})} \texttt{Order} \right) \bigg)
$$

1. Rename \texttt{cID} and \texttt{oID} in the first copy of \texttt{Order} to \texttt{cID1} and \texttt{oID1}, and in the second copy to \texttt{cID2} and \texttt{oID2}. This isolates \texttt{pID} as the only common attribute for the join.

$$
    \rho_{(\texttt{cID} \rightarrow \texttt{cID1}, \texttt{oID} \rightarrow \texttt{oID1})} \texttt{Order}, \quad \rho_{(\texttt{cID} \rightarrow \texttt{cID2}, \texttt{oID} \rightarrow \texttt{oID2})} \texttt{Order}
$$

2. Perform a natural join between the two renamed relations. Since \texttt{pID} is the only common attribute, the join will only keep rows where the \texttt{pID} values match.

$$
    \rho_{(\texttt{cID} \rightarrow \texttt{cID1}, \texttt{oID} \rightarrow \texttt{oID1})} \texttt{Order} \bowtie \rho_{(\texttt{cID} \rightarrow \texttt{cID2}, \texttt{oID} \rightarrow \texttt{oID2})} \texttt{Order}
$$

3. Use a selection condition to filter rows where $\texttt{cID1} \neq \texttt{cID2}$, ensuring that only pairs of distinct customers are included.

$$
    \sigma_{\texttt{cID1} \neq \texttt{cID2}}
$$

4. Project the final result to include only the columns \texttt{cID1} and \texttt{cID2}.

$$
    \pi_{\texttt{cID1, cID2}}
$$

\newpage
\subsection*{(g)}
Let $ A $ be the projection of $ R $ on $ A_1, \ldots, A_{k_1} $ such that $T\subseteq A$:
$$
    A := \pi_{A_1, \ldots, A_{k_1}}(R)
$$

Compute the set of tuples in $ A \times S $ that are not in $ R $ ($t\cdot s\notin R, t\in A$):
$$
    M := (A \times S) - R
$$

Project the invalid tuples (those in $ A $) from $ M $:
$$
    M_A := \pi_{A_1, \ldots, A_{k_1}}(M)
$$

Subtract the invalid tuples $ M_T $ from $ T $ to get the final result:
$$
    T := A - M_A
$$

Final Relational Algebra Expression:
$$
    R \div S := \pi_{A_1, \ldots, A_{k_1}}(R) - \pi_{A_1, \ldots, A_{k_1}} \bigg( ( \pi_{A_1, \ldots, A_{k_1}}(R) \times S ) - R \bigg)
$$

\newpage
\section*{Question 2}
\subsection*{(a)}
\textbf{F}, the \texttt{Prodcut} relation has an designated primary key of \texttt{\{maker, model, type\}}. If \texttt{\{maker, type\}} is a valid candidate key, \texttt{\{maker, model, type\}} would no longer be minimal and could not be the primary key, leading to a contradiction. Hence, \texttt{\{maker, type\}} is not a candidate key.

\subsection*{(b)}
\textbf{T}, \texttt{\{model, speed\}} is a superkey for the Laptop relation because it contains the primary key \texttt{\{model\}}, which is sufficient to uniquely identify rows.

\subsection*{(c)}
\textbf{T}, because the \texttt{Product} table schema specifies that the primary key is \texttt{\{maker, model, type\}}. This means that the combination of the values in the \texttt{maker}, \texttt{model}, and \texttt{type} columns must be unique across all rows in the \texttt{Product} table. The table already has a row (A, 001, pc''), then trying to insert another row with the same values (A, 001, pc'') would violate the uniqueness constraint of the primary key.

\subsection*{(d)}
\textbf{F}, it will violates the constraint of the models for all the pc products in the \texttt{Product} relation must be contained in the \texttt{PC} relation.

\subsection*{(e)}
\textbf{T}, because each subset of \texttt{\{color, type, price\}} combined with \texttt{\{model\}} forms a unique superkey. \texttt{\{color, type, price\}} has $2^3=8$ subsets.

\newpage
\section*{Question 3}
\subsection*{(a)}
First, select laptops with \texttt{speed} $\geq 2.3$:
$$
    \sigma_{\texttt{speed} \geq 2.3}(\texttt{Laptop})
$$

Then, join the result with the \texttt{Product} table on \texttt{model} and \texttt{type} to find the corresponding manufacturer (\texttt{maker}):
$$
    \sigma_{\texttt{speed} \geq 2.3}(\texttt{Laptop}) \bowtie_{\texttt{Laptop.model=Product.model}\land\texttt{type=}'laptop'} \texttt{Product}
$$

Finally, project the \texttt{maker} attribute to get the final result:
$$
    \pi_{\texttt{maker}} \bigg( \sigma_{\texttt{speed} \geq 2.3}(\texttt{Laptop}) \bowtie_{\texttt{Laptop.model=Product.model}\land\texttt{type=}'laptop'} \texttt{Product} \bigg)
$$

\subsection*{(b)}
1. Select rows in \texttt{Product} where \texttt{maker} = 'D':
$$
    \sigma_{\texttt{maker} = 'D'}(\texttt{Product})
$$

2. Perform theta joins with type-specific conditions:

- For \texttt{PC}:
$$
    \pi_{\texttt{model, type, price}} \Big(
    (\sigma_{\texttt{maker} = 'D'}(\texttt{Product}) \bowtie_{\texttt{Product.model} = \texttt{PC.model} \land \texttt{type} = 'pc'} \texttt{PC})
    \Big)
$$
- For \texttt{Laptop}:
$$
    \pi_{\texttt{model, type, price}} \Big(
    (\sigma_{\texttt{maker} = 'D'}(\texttt{Product}) \bowtie_{\texttt{Product.model} = \texttt{Laptop.model} \land \texttt{type} = 'laptop'} \texttt{Laptop})
    \Big)
$$
- For \texttt{Printer}:
$$
    \pi_{\texttt{model, type, price}} \Big(
    (\sigma_{\texttt{maker} = 'D'}(\texttt{Product}) \bowtie_{\texttt{Product.model} = \texttt{Printer.model} \land \texttt{type} = 'printer'} \texttt{Printer})
    \Big)
$$

3. Union the projected results:
$$
    \begin{aligned}
             & \Big(\pi_{\texttt{model, type, price}}(\sigma_{\texttt{maker} = 'D'}(\texttt{Product}) \bowtie_{\texttt{Product.model} = \texttt{PC.model} \land \texttt{type} = 'pc'} \texttt{PC})
        \Big)                                                                                                                                                                                      \\
        \cup &
        \Big(
        \pi_{\texttt{model, type, price}}(\sigma_{\texttt{maker} = 'D'}(\texttt{Product}) \bowtie_{\texttt{Product.model} = \texttt{Laptop.model} \land \texttt{type} = 'laptop'} \texttt{Laptop})
        \Big)                                                                                                                                                                                      \\
        \cup &
        \Big(
        \pi_{\texttt{model, type, price}}(\sigma_{\texttt{maker} = 'D'}(\texttt{Product}) \bowtie_{\texttt{Product.model} = \texttt{Printer.model} \land \texttt{type} = 'printer'} \texttt{Printer})
        \Big)
    \end{aligned}
$$

\subsection*{(c)}
1. Get two copies of \texttt{PC} and rename the \texttt{model} attribute to distinguish them in the final output:
$$
    \texttt{PC\_1}\leftarrow\rho_{(\texttt{PC\_1},\texttt{model}\rightarrow \texttt{model\_1})}\texttt{PC}
$$
$$
    \texttt{PC\_2}\leftarrow\rho_{(\texttt{PC\_2},\texttt{model}\rightarrow \texttt{model\_2})}\texttt{PC}
$$
2. Self-join the \texttt{PC} relations on \texttt{speed} and \texttt{ram} (using \texttt{model\_1} $<$ \texttt{model\_2} to ensure every pair is unique and their elements are distinct):
$$
    \texttt{PC\_1} \bowtie_{(\texttt{PC\_1.speed} = \texttt{PC\_2.speed} \land \texttt{PC\_1.ram} = \texttt{PC\_2.ram} \land \texttt{model\_1} < \texttt{model\_2})} \texttt{PC\_2}
$$

3. Project the relevant attributes (\texttt{model\_1}, \texttt{model\_2}):
$$
    \pi_{(\texttt{model\_1}, \texttt{model\_2})} \Big(\texttt{PC\_1} \bowtie_{(\texttt{PC\_1.speed} = \texttt{PC\_2.speed} \land \texttt{PC\_1.ram} = \texttt{PC\_2.ram} \land \texttt{model\_1} < \texttt{model\_2})} \texttt{PC\_2}
    \Big)
$$

\newpage
\subsection*{(d)}
1. Self-join the \texttt{Printer} table to compare prices and exclude printers that are not the cheapest.
$$
    \texttt{ExpensivePrinters} \leftarrow \pi_{\texttt{p1.model}} \Big(
    \rho_{\texttt{p1}}(\texttt{Printer}) \bowtie_{\texttt{p1.price} > \texttt{p2.price}} \rho_{\texttt{p2}}(\texttt{Printer})
    \Big)
$$

2. Find printers that are not in \texttt{ExpensivePrinters} to get the models of the cheapest printers:
$$
    \texttt{CheapestPrinters} \leftarrow \pi_{\texttt{model}}(\texttt{Printer}) - \texttt{ExpensivePrinters}
$$

3. Select printers:
$$
    \texttt{PrinterProducts} \leftarrow \sigma_{\texttt{type} = 'printer'}(\texttt{Product})
$$

4. Natural join with the \texttt{PrinterProducts} table to get the manufacturers:
$$
    \texttt{CheapestPrinterProducts} \leftarrow \texttt{CheapestPrinters} \bowtie \texttt{PrinterProducts}
$$

5. Project the \texttt{maker} attribute to get the final result:
$$
    \pi_{\texttt{maker}}(\texttt{CheapestPrinterProducts})
$$

\subsection*{(e)}
1. Select printers:
$$
    \texttt{PrinterProducts} \leftarrow \sigma_{\texttt{type} = 'printer'}(\texttt{Product})
$$

2. Self-join to find products with at least 2 distinct manufacturers:
$$
    \texttt{Pairs} \leftarrow \rho_{\texttt{P1}}(\texttt{PrinterProducts}) \bowtie_{\texttt{P1.model} = \texttt{P2.model} \land \texttt{P1.maker} < \texttt{P2.maker}} \rho_{\texttt{P2}}(\texttt{PrinterProducts})
$$
3. Project the `\texttt{model}` to identify products with at least 2 distinct manufacturers:
$$
    \texttt{ModelsWithTwoMakers} \leftarrow \pi_{\texttt{P1.model}}(\texttt{Pairs})
$$
4. Perform a triple self-join on `\texttt{PrinterProducts}` as we need to find sets of 3 distinct manufacturers for the same `\texttt{model}`. This requires joining 3 instances of the `\texttt{PrinterProducts}` table:
$$
    \begin{aligned}
        \texttt{TripleJoin} \leftarrow
         & \rho_{\texttt{P1}}(\texttt{PrinterProducts})
        \bowtie_{\texttt{P1.model} = \texttt{P2.model} \land \texttt{P1.maker} < \texttt{P2.maker}} \\
         & \rho_{\texttt{P2}}(\texttt{PrinterProducts})
        \bowtie_{\texttt{P2.model} = \texttt{P3.model} \land \texttt{P2.maker} < \texttt{P3.maker}} \\
         & \rho_{\texttt{P3}}(\texttt{PrinterProducts})
    \end{aligned}
$$

5. Project the `\texttt{model}` to identify products with at least 3 distinct manufacturers:
$$
    \texttt{ModelsWithThreeMakers} \leftarrow \pi_{\texttt{P1.model}}(\texttt{TripleJoin})
$$
6. Remove models with at least 3 manufacturers from those with at least 2 manufacturers and get models with exactly 2 manufacturers:
$$
    \texttt{ExactlyTwoMakers} \leftarrow \texttt{ModelsWithTwoMakers} - \texttt{ModelsWithThreeMakers}
$$

\end{document}
