\documentclass{article}
\usepackage[a4paper,margin=1in]{geometry}
\usepackage{amsmath,amsfonts,amssymb,graphicx}
\usepackage{listings}
\usepackage{xcolor}

\definecolor{codegray}{gray}{0.9}
\definecolor{darkgreen}{rgb}{0,0.5,0}
\definecolor{darkblue}{rgb}{0,0,0.6}
\lstdefinelanguage{SQL}{
  keywords={SELECT, FROM, WHERE, JOIN, ON, AS, GROUP, BY, ORDER, HAVING, INSERT, INTO, UPDATE, DELETE, CREATE, TABLE, DROP, ALTER, ADD, PRIMARY, FOREIGN, KEY, CONSTRAINT, CHECK, DEFAULT, UNION, ALL, DISTINCT, WITH, RECURSIVE, LIMIT, DESC, ASC},
  keywordstyle=\color{darkblue}\bfseries,
  morekeywords={[2]{COUNT, SUM, AVG, MIN, MAX}},
  keywordstyle={[2]\color{darkgreen}\bfseries},
  identifierstyle=\color{black},
  sensitive=false,
  comment=[l]{--},
  morecomment=[s]{/*}{*/},
  commentstyle=\color{gray}\itshape,
  stringstyle=\color{red},
  morestring=[b]',
  morestring=[b]"
}

\lstset{
  language=SQL,
  backgroundcolor=\color{codegray},
  basicstyle=\ttfamily\footnotesize,
  columns=fullflexible,
  breaklines=true,
  keepspaces=true,
  showstringspaces=false,
  frame=single,
  rulecolor=\color{black}
}
\title{\textbf{CS 348 Winter 2025: Assignment 2 - Part I}}
\author{Jiaze Xiao \\ 20933691}
\date{\today}
\setlength{\parindent}{0pt}
\begin{document}

\maketitle

\section*{Question 1}
\begin{lstlisting}
    WITH RECURSIVE ShortestPaths(dst, cost) AS (
        -- Only keep the shortest paths from Waterloo to each (reachable) destination
        SELECT dst, MIN(cost)
        FROM (
            -- Base case: Direct flights from Waterloo
            SELECT dst, cost
            FROM Flights
            WHERE src = 'Waterloo'

            UNION

            -- Recursive case: Extend paths from Waterloo with i edges
            SELECT f.dst, sp.cost + f.cost as cost
            FROM ShortestPaths sp
            JOIN Flights f ON sp.dst = f.src
        )
        GROUP BY dst
    )
    SELECT * FROM ShortestPaths;
\end{lstlisting}
My solution starts with all direct flights from ``Waterloo''. At each recursive step, expands paths while only keeping the shortest path per destination at each step. The \texttt{MIN(cost) GROUP BY dst} inside the recursive CTE eliminates unnecessary longer paths immediately, keeping the intermediate relation at most size $n$. Since we extend by one edge per step and the graph is guaranteed a DAG, the recursion stops in at most $d+1$ iterations.

\newpage
\section*{Question 2}
\subsection*{(a)}
DDL Query for \texttt{SaleDetails} Table:
\begin{lstlisting}
  CREATE TABLE SaleDetails (
      ISBN VARCHAR(30),
      sid INT,
      quantity INT,
      PRIMARY KEY (ISBN, sid),
      FOREIGN KEY (ISBN) REFERENCES Book,
      FOREIGN KEY (sid) REFERENCES Sale
  );
\end{lstlisting}
Adding Constraint for Quantity After Table Creation:
\begin{lstlisting}
  ALTER TABLE SaleDetails
  ADD CONSTRAINT quantity_check CHECK (quantity >= 1 AND quantity <= 100);
\end{lstlisting}

\newpage
\subsection*{(b)}
SQL Query for the \texttt{Master} Relation:
\begin{lstlisting}
  SELECT *
  FROM (
    SELECT
      A.aid,
      A.name AS author_name,
      B.ISBN,
      B.name AS book_name,
      B.copies AS book_copies
    FROM Author A
    JOIN Writes W ON A.aid = W.aid
    JOIN Book B ON W.ISBN = B.ISBN
    WHERE B.copies > 0

    UNION

    SELECT
      A1.aid,
      A1.name AS author_name,
      NULL AS ISBN,
      NULL AS book_name,
      NULL AS book_copies
    FROM Author A1
    WHERE NOT EXISTS (
      SELECT 1
      FROM Author A2
      JOIN Writes W ON A2.aid = W.aid
      JOIN Book B ON W.ISBN = B.ISBN
      WHERE B.copies > 0 AND A1.aid = A2.aid
    )

    UNION

    SELECT
      NULL AS aid,
      NULL AS author_name,
      B.ISBN,
      B.name AS book_name,
      B.copies AS book_copies
    FROM Book B
    LEFT JOIN Writes W ON B.ISBN = W.ISBN
    LEFT JOIN Author A ON A.aid = W.aid
    WHERE (A.aid IS NULL OR A.name IS NULL) AND B.copies > 0
  );
\end{lstlisting}

\newpage
\subsection*{(c)}
\begin{lstlisting}
  WITH PublisherSales AS (
      SELECT
          B.pid,
          B.ISBN,
          SUM(SD.quantity) AS total_quantity
      FROM SaleDetails SD
      JOIN Book B ON SD.ISBN = B.ISBN
      GROUP BY B.pid, B.ISBN
  ),
  FilteredPublishers AS (
      SELECT pid
      FROM PublisherSales
      GROUP BY pid
      HAVING COUNT(DISTINCT ISBN) >= 2
  ),
  PublisherHighestSales AS (
      SELECT PS.pid, PS.ISBN, PS.total_quantity
      FROM PublisherSales PS
      JOIN FilteredPublishers FP ON PS.pid = FP.pid
      WHERE PS.ISBN = (
          SELECT PS2.ISBN
          FROM PublisherSales PS2
          WHERE PS2.pid = PS.pid
          ORDER BY PS2.total_quantity DESC
          LIMIT 1
      )
  )
  SELECT B.name, PHS.total_quantity
  FROM PublisherHighestSales PHS
  JOIN Book B ON PHS.ISBN = B.ISBN;
\end{lstlisting}

\newpage
\section*{Question 3}
\subsection*{(a)}
In a binary relation $R(A,B)$, the only possible non-trivial functional dependencies are:
\begin{itemize}
  \item $ A \rightarrow B $
  \item $ B \rightarrow A $
\end{itemize}
Since $ R $ has only two attributes, any non-trivial functional dependency must determine all attributes other than those on the left-hand side. For example:
\begin{itemize}
  \item If $ A \to B $, then $ A $ must be a key because it determines all other attributes in $ R $.
  \item Similarly, if $ B \to A $, then $ B $ must be a key.
\end{itemize}

In either case, the determinant of the functional dependency is a superkey. Therefore, $ R(A, B) $ satisfies the BCNF condition.

\subsection*{(b)}
\subsubsection*{(i)}
Use the \texttt{AttributeClosure} algorithm to compute $ A^+ $:

1. Start with $ \text{result} = \{A\} $.

2. Iterate through the functional dependencies:
\begin{itemize}
  \item $ fd_3: A \to D $: Add $ D $ to $ \text{result} $. Now $ \text{result} = \{A, D\} $.
  \item No other functional dependencies have $ A $ or $ D $ on the left-hand side.

\end{itemize}
Final result: $ A^+ = \{A, D\} $.

\subsubsection*{(ii)}
Use the \texttt{AttributeClosure} algorithm to compute $ E^+ $:

1. Start with $ \text{result} = \{E\} $.

2. Iterate through the functional dependencies:
\begin{itemize}
  \item $ fd_1: E \to AB $: Add $ A $ and $ B $ to $ \text{result} $. Now $ \text{result} = \{E, A, B\} $.
  \item $ fd_3: A \to D $: Add $ D $ to $ \text{result} $. Now $ \text{result} = \{E, A, B, D\} $.
  \item $ fd_2: BD \to C $: Now $ \text{result} = \{E, A, B, D, C\} $.
  \item All attributes are in result, no more attributes will be added.
\end{itemize}
Final result: $ E^+ = \{A, B, C, D, E\} $.

\subsubsection*{(iii)}
By definition, $X^+$ contains all attributes that are functionally dependent on $X$. Thus, if $X^+$ contains all attributes, $X$ can determine all attributes uniquely which means $X$ can be a primary key.
\begin{itemize}
  \item $ A^+ = \{A, D\} $: Does not include all attributes ($ B, C, E $ are missing). Thus, $ A $ cannot be a primary key.
  \item $ E^+ = \{A, B, C, D, E\} $: Includes all attributes. Thus, $ E $ can be a primary key.
\end{itemize}

\newpage
\section*{Question 4}
\subsection*{(a)}
Single attributes:
\begin{itemize}
  \item $ A^+ = \{A\} $ (no FDs directly involve $ A $ alone).
  \item $ B^+ = \{B\} $ (no FDs directly involve $ B $ alone).
  \item $ C^+ = \{C\} $ (no FDs directly involve $ C $ alone).
  \item $ D^+ = \{D\} $ (no FDs directly involve $ D $ alone).
  \item $ E^+ = \{E\} $ (no FDs directly involve $ E $ alone).
  \item $ F^+ = \{F\} $ (no FDs directly involve $ F $ alone).
\end{itemize}

Two-attribute combinations:
\begin{itemize}
  \item $ AB^+ = \{A, B, C, E\} $ (using $ AB \to CE $).
        \begin{itemize}
          \item Add $ D $ using $ AC \to DE $: $ AB^+ = \{A, B, C, D, E\} $.
          \item Add $ F $ using $ BE \to CF $: $ AB^+ = \{A, B, C, D, E, F\} $.
          \item $ AB $ determines all attributes, so $ AB $ is a candidate key.
        \end{itemize}
  \item $ AC^+ = \{A, C, D, E\} $ (using $ AC \to DE $).
        \begin{itemize}
          \item Add $ B $ using $ ACD \to B $: $ AC^+ = \{A, B, C, D, E\} $.
          \item Add $ F $ using $ BE \to CF $: $ AC^+ = \{A, B, C, D, E, F\} $.
          \item $ AC $ determines all attributes, so $ AC $ is a candidate key.
        \end{itemize}
  \item $ BF^+ = \{B, F, A\} $ (using $ BF \to A $).
        \begin{itemize}
          \item Add $ C $ using $ AB \to CE $: $ BF^+ = \{A, B, C, E, F\} $.
          \item Add $ D $ using $ AC \to DE $: $ BF^+ = \{A, B, C, D, E, F\} $.
          \item $ BF $ determines all attributes, so $ BF $ is a candidate key.
        \end{itemize}
  \item $ BE^+ = \{B, E, C, F\} $ (using $ BE \to CF $).
        \begin{itemize}
          \item Add $ A $ using $ BF \to A $: $ BE^+ = \{A, B, C, E, F\} $.
          \item Add $ D $ using $ AC \to DE $: $ BE^+ = \{A, B, C, D, E, F\} $.
          \item $ BE $ determines all attributes, so $ BE $ is a candidate key.
        \end{itemize}
\end{itemize}
There is no other minimal key. Thus, the candidate keys are:
$$
  \boxed{\{AB, AC, BF, BE\}}
$$
\subsection*{(b)}
Split RHS into Single Attributes:
$$
  F = \{ AB \to C, AB \to E, ACD \to B, BE \to C, BE \to F, AC \to D, AC \to E, BF \to A \}.
$$
Simplify LHS:
\begin{itemize}
  \item $ACD \to B$ can be simplified to $AC \to B$ since $AC \to D$, $D$ is not neseccery in the dependency.
\end{itemize}
Remove Redundant FDs:
\begin{itemize}
  \item $ BE \to C $: From $ BE \to F $, $ BF \to A $, and $ AB \to C $, we can derive $ BE \to C $. Thus, $ BE \to C $ is redundant and can be removed.
  \item $ AB \to E $: From $ AC \to E $ and $ AB \to C $, we can derive $ AB \to E $. Thus, $ AB \to E $ is redundant and can be removed.
\end{itemize}
After removing the redundant FDs ($ BE \to C $, and $ AB \to E $), the minimal cover becomes:
$$
  \boxed{F_{\text{min}} = \{ AB \to C, AC \to B, BE \to F, AC \to D, AC \to E, BF \to A \}}
$$
\newpage
\subsection*{(c)}
\subsubsection*{$AB\to E$:}
In $R_1$, the fd $AB\to C$ is preserved.\\
In $R_3$, the fd $AC\to E$ is preserved.\\
We can derive the fd $AB\to E$ from them. Thus, $AB\to E$ is preserved.
\subsubsection*{$BE\to F$:}
Since $B$ and $E$ are split across different relations, it is impossible to preserve $BE\to F$.

\subsection*{(d)}
$$A^+=\{A,B\},\quad ED^+=\{A,B,D,E\}$$
Since $A$ is not a superkey of $R_2$ and $A\to B$ is not trivial, $R_2$ violates $A\to B$. $R_2$ will be decompsed to:
$$AB\quad\text{and}\quad ACDE$$
Since $ED$ is not a super key of $ACDE$ and $ED\to A$ is not trivial. $ACDE$ will be decompsed to:
$$ADE\quad\text{and}\quad CDE$$
As a \texttt{BCNF} lossless-join decomposition of $R_2$ is:
$$\boxed{\{AB, ADE, CDE\}}$$
\newpage
\end{document}
